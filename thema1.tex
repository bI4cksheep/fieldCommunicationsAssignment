\section{Vorteile der Gleichanteilsfreiheit bei der binären Kodierung}
Die binäre Kodierung ist eine Abbildungsvorschrift, welche Nullen und Einsen verwendet,
um Informationen darzustellen. Jedes Wort dieser Sprache kann über dem Alphabet {0,1} gebildet werden.
Sie ist besonders in der Digitaltechnik relevant, da die zwei möglichen Zeichen durch
einen High- und Low-Zustand abgebildet werden können, sodass komplizierte Messschaltungen entfallen. \autocite{Fricke2003}
Jede Art von digitalen Daten (Fotos, Videos, Dokumente, etc.) lässt sich mithilfe dieser Kodierung
darstellen, speichern und übertragen.
Bei der digitalen, binären Informationsübertragung, z.B. über einen Datenbus, muss die binäre Kodierung erneut
codiert werden, beispielweise indem der Null und der Eins jeweils eine bestimmte Amplitude eines 
Signals zugewiesen wird. \\
Es sind verschiedene digitale Codierungsarten möglich, von denen sich jedoch nicht jede für jeden Einsatzzweck eignet.
Oft bedeutet die Übertragung von digital codierten Signalen, dass zusätzlich zu den Leitungen der Energieversorgung auch 
Datenleitungen verlegt werden müssen, wodurch der Verkabelungsaufwand ansteigt. Eine alternative Möglichkeit bietet sich 
in der Signalübertragung über die Stromleitungen selbst. Dies ist möglich, wenn die Komponenten mit Gleichspannung versorgt werden.
Voraussetzung für diese Art der Datenübertragung ist, dass die Signale gleichanteilsfrei sind.
Unter dem Gleichanteil eines Signals versteht man den Langzeitmittelwert dieses Signals. Ist ein Signal gleichanteilsfrei, dann
weißt es einen Gleichanteil von 0 auf.
