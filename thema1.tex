\section{Vorteile der Gleichanteilsfreiheit bei der binären Codierung}
Die binäre Kodierung ist eine Abbildungsvorschrift, welche Nullen und Einsen verwendet
um Informationen darzustellen. Jedes Wort dieser Sprache kann über dem Alphabet \{0,1\} gebildet werden.
Sie ist besonders in der Digitaltechnik relevant, da die zwei möglichen Zeichen durch
einen High- und Low-Zustand (z.B. Spannung an, Spannung aus) abgebildet werden können, sodass komplizierte Messschaltungen entfallen.\autocite[379]{Fricke2003}
Jede Art von digitalen Daten (Fotos, Videos, Dokumente, etc.) lässt sich mithilfe dieser Kodierung
darstellen, speichern und übertragen. Um beispielsweise Text darzustellen, können Buchstaben und Satzzeichen
durch den 7-Bit ASCII Code dargestellt werden.
Sollen diese Binärdaten nun digital übertragen werden, z.B. über einen Datenbus, müssen die Einsen und Nullen
der binären Codierung erneut codiert werden. Hierfür eignet es sich unter Anderem, der Eins und der Null
verschiedene Amplituden eines Signals zuzuweisen.\\
Es sind verschiedene digitale Codierungsarten möglich, von denen sich jedoch nicht jede für jeden Einsatzzweck eignet.
Manche Codierungen bieten die Möglichkeit, Daten synchron zu übertragen, andere wiederum übertragen die Daten asynchron.
Oft bedeutet die Übertragung von digital codierten Signalen aber auch, dass zusätzlich zu den Leitungen der Energieversorgung auch 
Datenleitungen verlegt werden müssen, sodass der Verkabelungsaufwand ansteigt. Eine alternative Möglichkeit bietet sich 
in der Verwendung einer Codierung, welche die Signalübertragung über die Stromleitungen selbst ermöglicht.
Für diese Art der Signalübertragung gilt es, zwei Voraussetzungen zu erfüllen. Zunächst muss die Energieversorgung der
Komponenten mittels Gleichspannung erfolgen. Außerdem wird die Anforderung der Gleichanteilsfreiheit an die übertragenen
Signale gestellt. \autocite[55]{Schnell2019}
Unter dem Gleichanteil eines Signals versteht man den Langzeitmittelwert dieses Signals. Ist ein Signal gleichanteilsfrei, dann
weißt es einen Gleichanteil von 0 auf, weicht also im Mittel werde positiv noch negativ ab. \\

