\section{Das ist ein Test}
Die binäre Kodierung ist eine Abbildungsvorschrift, welche Nullen und Einsen verwendet,
um Informationen darzustellen. Jedes Wort dieser Sprache kann über dem Alphabet {0,1} gebildet werden.
Sie ist besonders in der Digitaltechnik relevant, da die zwei möglichen Zeichen durch
einen High- und Low-Zustand abgebildet werden können, sodass komplizierte Messschaltungen entfallen. \autocite{Fricke2003}
Jede Art von digitalen Daten (Fotos, Videos, Dokumente, etc.) lässt sich mithilfe dieser Kodierung
darstellen, speichern und übertragen.
Bei der binären Informationsübertragung, z.B. über einen Datenbus, muss die binäre Kodierung erneut
codiert werden, beispielweise indem der Null und der Eins jeweils eine bestimmte Amplitude eines 
Signals zugewiesen wird.
